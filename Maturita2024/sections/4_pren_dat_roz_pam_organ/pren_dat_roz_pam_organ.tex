\section{Přenos dat, rozdělení pamětí a jejich organizace}
\subsection{Sériový a paralelní přenos}
\paragraph{Sériový přenos}
\begin{itemize}
    \item \textbf{Definice}: Sériový přenos je metoda, při které jsou data přenášena bit po bitu po jedné lince nebo kanálu.
    \item \textbf{Vlastnosti}:
    \begin{itemize}
        \item \textbf{Jedna linka}: Využívá pouze jednu linku pro přenos dat.
        \item \textbf{Rychlost}: Sériový přenos může být pomalejší pro krátké vzdálenosti, ale je efektivnější pro delší vzdálenosti.
        \item \textbf{Příklady}: USB, RS-232, RS-485, I2C, SPI.
        \item \textbf{Výhody}: Menší rušení, menší počet vodičů, jednodušší kabeláž.
        \item \textbf{Nevýhody}: Nižší rychlost přenosu pro krátké vzdálenosti ve srovnání s paralelním přenosem.
    \end{itemize}
\end{itemize}

\paragraph{Paralelní přenos}
\begin{itemize}
    \item \textbf{Definice}: Paralelní přenos je metoda, při které jsou data přenášena po několika linkách současně.
    \item \textbf{Vlastnosti}:
    \begin{itemize}
        \item \textbf{Více linek}: Využívá více linek (obvykle tolik, kolik bitů je v jedné datové jednotce, například 8, 16, nebo 32).
        \item \textbf{Rychlost}: Vysoká rychlost přenosu pro krátké vzdálenosti, protože více bitů je přenášeno současně.
        \item \textbf{Příklady}: Starší počítačové sběrnice jako PCI, starší tiskárny používající paralelní porty.
        \item \textbf{Výhody}: Vysoká přenosová rychlost pro krátké vzdálenosti.
        \item \textbf{Nevýhody}: Vyšší nároky na kabeláž (více vodičů), větší rušení, problém s synchronizací dat na delší vzdálenosti.
    \end{itemize}
\end{itemize}

%včetně vytvoření sběrnice?
\paragraph{Připojení s otevřeným kolektorem}
\begin{itemize}
    \item \textbf{Definice}: Připojení s otevřeným kolektorem je způsob připojení, kde je výstupní tranzistor obvodu připojený ke kolektoru, který není přímo spojen s napájecím napětím, ale je dostupný externímu obvodu.
    \item \textbf{Princip fungování}:
    \begin{itemize}
        \item Tranzistor se otevře (zapne), když je na bázi přítomen dostatečný proud, což umožní proud proudit z kolektoru do emitoru.
        \item Když je tranzistor vypnutý, kolektor je v podstatě odpojen od obvodu, což umožňuje použít externí pull-up rezistor k napájení kolektoru.
    \end{itemize}
    \item \textbf{Výhody}:
    \begin{itemize}
        \item Možnost snadného připojení více výstupů na jednu sběrnici (wire-AND logika).
        \item Vysoká odolnost proti rušení.
    \end{itemize}
    \item \textbf{Nevýhody}:
    \begin{itemize}
        \item Potřeba externího pull-up rezistoru.
        \item Pomalejší rychlost přepínání kvůli kapacitě a hodnotě pull-up rezistoru.
    \end{itemize}
    \item \textbf{Příklady použití}: Komunikační sběrnice jako I2C, logické brány v digitálních obvodech.
\end{itemize}

\paragraph{Třístavový zesilovač}
\begin{itemize}
    \item \textbf{Definice}: Třístavový zesilovač je typ logického obvodu, který může být v jednom ze tří stavů: logická 1, logická 0, nebo vysoká impedance (odpojený stav).
    \item \textbf{Princip fungování}:
    \begin{itemize}
        \item Kromě vstupů pro logickou hodnotu (0 nebo 1) má třístavový zesilovač také vstup pro řízení povolení (enable).
        \item Když je povolovací vstup aktivní, zesilovač předává vstupní signál na výstup.
        \item Když je povolovací vstup neaktivní, výstup je odpojen (vysoká impedance), což umožňuje sdílet výstupní linku s jinými obvody.
    \end{itemize}
    \item \textbf{Výhody}:
    \begin{itemize}
        \item Umožňuje sdílení jedné výstupní linky více zařízeními bez rizika konfliktu signálů.
        \item Flexibilita v návrhu digitálních systémů.
    \end{itemize}
    \item \textbf{Nevýhody}:
    \begin{itemize}
        \item Komplexnější řízení, aby nedošlo ke konfliktu signálů.
    \end{itemize}
    \item \textbf{Příklady použití}: Datové sběrnice (např. systémová sběrnice v počítačích), paměťové moduly, mikroprocesorové systémy.
\end{itemize}

\subsection{Popis vybraných sběrnic}
\paragraph{$I^2$C}
\begin{itemize}
    \item \textbf{Definice}: I2C (Inter-Integrated Circuit) je víceúčelová sériová sběrnice, kterou vyvinula společnost Philips.
    \item \textbf{Vlastnosti}:
    \begin{itemize}
        \item Half-duplex.
        \item Používá dvě hlavní linie: SDA (Serial Data Line) a SCL (Serial Clock Line).
        \item Umožňuje připojení více master a slave zařízení na jednu sběrnici.
        \item Podporuje adresování až 128 zařízení na jedné sběrnici.
    \end{itemize}
    \item \textbf{Výhody}: Jednoduchost, nízké nároky na kabeláž, podpora více zařízení.
    \item \textbf{Nevýhody}: Relativně nízká rychlost přenosu, omezená délka sběrnice.
    \item \textbf{Příklady použití}: Senzory, displeje, paměťové moduly.
\end{itemize}

\paragraph{SPI}
\begin{itemize}
    \item \textbf{Definice}: SPI (Serial Peripheral Interface) je synchronní sériová komunikace vyvinutá společností Motorola.
    \item \textbf{Vlastnosti}:
    \begin{itemize}
        \item Používá čtyři hlavní linie: SCLK (Serial Clock), MOSI (Master Out Slave In), MISO (Master In Slave Out), SS (Slave Select).
        \item Full-duplex přenos dat.
        \item Rychlá komunikace díky synchronnímu přenosu.
    \end{itemize}
    \item \textbf{Výhody}: Vysoká rychlost, jednoduchá implementace, full-duplex přenos.
    \item \textbf{Nevýhody}: Vyšší počet vodičů, omezený počet připojených zařízení.
    \item \textbf{Příklady použití}: Paměťové čipy, senzory, displeje.
\end{itemize}

\paragraph{RS-232}
\begin{itemize}
    \item \textbf{Definice}: RS-232 (Recommended Standard 232) je standard pro sériovou komunikaci, vyvinutý asociací EIA.
    \item \textbf{Vlastnosti}:
    \begin{itemize}
        \item Asynchronní přenos dat.
        \item Používá signály jako TX (Transmit), RX (Receive), RTS (Request to Send), CTS (Clear to Send).
        \item Maximální délka kabelu až 15 metrů při nižších rychlostech.
    \end{itemize}
    \item \textbf{Výhody}: Jednoduchost, široká kompatibilita.
    \item \textbf{Nevýhody}: Omezená rychlost, citlivost na rušení.
    \item \textbf{Příklady použití}: Modemy, terminály, průmyslové zařízení.
\end{itemize}

\paragraph{IEEE 1284}
\begin{itemize}
    \item \textbf{Definice}: IEEE 1284 je standard pro paralelní komunikaci mezi počítačem a periferními zařízeními.
    \item \textbf{Vlastnosti}:
    \begin{itemize}
        \item Podporuje vysokorychlostní přenosy dat.
        \item Kompatibilní s původním Centronics paralelním portem.
        \item Více režimů přenosu (např. kompatibilní mód, nibble mód, ECP, EPP).
    \end{itemize}
    \item \textbf{Výhody}: Vysoká rychlost přenosu, robustnost.
    \item \textbf{Nevýhody}: Větší počet vodičů, omezená délka kabelu.
    \item \textbf{Příklady použití}: Tiskárny, skenery, externí disky.
\end{itemize}

\paragraph{USB}
\begin{itemize}
    \item \textbf{Definice}: USB (Universal Serial Bus) je standard pro připojení periferních zařízení k počítači.
    \item \textbf{Vlastnosti}:
    \begin{itemize}
        \item Používá sériovou komunikaci.
        \item Podpora Plug and Play.
        \item Možnost napájení připojených zařízení.
    \end{itemize}
    \item \textbf{Výhody}: Vysoká rychlost přenosu, jednoduchost připojení, široká kompatibilita.
    \item \textbf{Nevýhody}: Omezená délka kabelu, potřeba driverů.
    \item \textbf{Příklady použití}: Klávesnice, myši, tiskárny, externí disky, USB flash disky.
\end{itemize}

\subsection{Rozdělení pamětí v PC včetně jejich popisu}
\paragraph{RAM}
\begin{itemize}
    \item \textbf{Definice}: RAM (Random Access Memory) je typ paměti, která umožňuje přístup k jakékoli paměťové buňce přímo.
    \item \textbf{Vlastnosti}:
    \begin{itemize}
        \item Rychlý přístup k datům.
        \item Volatilní paměť (data se ztrácí při vypnutí napájení).
        \item Používá se jako operační paměť v počítačích.
    \end{itemize}
    \item \textbf{Typy}:
    \begin{itemize}
        \item DRAM (Dynamic RAM): Používá kondenzátory, které vyžadují pravidelné obnovování.
        \item SRAM (Static RAM): Používá bistabilní klopné obvody, nevyžaduje obnovování.
    \end{itemize}
\end{itemize}

\paragraph{ROM}
\begin{itemize}
    \item \textbf{Definice}: ROM (Read-Only Memory) je typ paměti, do které nelze běžně zapisovat data, ale lze je z ní pouze číst.
    \item \textbf{Vlastnosti}:
    \begin{itemize}
        \item Nevolatilní paměť (data se neztrácí při vypnutí napájení).
        \item Obsahuje firmware a základní programy pro spuštění počítače.
    \end{itemize}
    \item \textbf{Typy}:
    \begin{itemize}
        \item PROM (Programmable ROM): Paměť, kterou lze naprogramovat pouze jednou.
        \item EPROM (Erasable Programmable ROM): Paměť, kterou lze vymazat pomocí UV světla a znovu naprogramovat.
        \item EEPROM (Electrically Erasable Programmable ROM): Paměť, kterou lze elektricky vymazat a znovu naprogramovat.
    \end{itemize}
\end{itemize}

\subsection{Organizace paměti}
\paragraph{Kapacita}
\begin{itemize}
    \item \textbf{Definice}: Kapacita paměti udává celkové množství dat, které může paměť uchovat.
    \item \textbf{Jednotky}: Bity, byty, kilobyty (KB), megabyty (MB), gigabyty (GB), terabyty (TB).
\end{itemize}

\paragraph{Hloubka}
\begin{itemize}
    \item \textbf{Definice}: Hloubka paměti udává počet adresovatelných buněk v paměti.
    \item \textbf{Výpočet}: Hloubka paměti se často udává jako 2 na n-tou, kde n je počet adresních linek.
\end{itemize}

\paragraph{Délka datového slova}
\begin{itemize}
    \item \textbf{Definice}: Délka datového slova udává počet bitů, které mohou být současně zapsány nebo přečteny z paměti.
    \item \textbf{Příklady}: 8 bitů, 16 bitů, 32 bitů, 64 bitů.
\end{itemize}

\paragraph{Paměťová mapa}
\begin{itemize}
    \item \textbf{Definice}: Paměťová mapa je schéma, které ukazuje rozložení paměťových oblastí v adresním prostoru počítače nebo mikrokontroléru.
    \item \textbf{Význam}: 
    \begin{itemize}
        \item Umožňuje efektivní organizaci a přístup k různým typům paměti, jako jsou RAM, ROM, I/O porty a speciální registry.
        \item Pomáhá při ladění a optimalizaci programů, protože programátor může přesně určit, kde se nachází jednotlivé proměnné a kódy.
    \end{itemize}
    \item \textbf{Příklady použití}:
    \begin{itemize}
        \item \textbf{ROM oblast}: Obsahuje firmware a základní vstupně-výstupní systém (BIOS).
        \item \textbf{RAM oblast}: Používá se pro dynamická data a běžící programy.
        \item \textbf{I/O oblast}: Prostor vyhrazený pro komunikaci s periferiemi jako jsou klávesnice, myši, tiskárny.
        \item \textbf{Speciální registry}: Používají se pro řízení hardwaru, jako jsou časovače a řadiče přerušení.
    \end{itemize}
    \item \textbf{Příklad paměťové mapy}:
    \begin{itemize}
        \item \textbf{0x0000 - 0x7FFF}: ROM oblast
        \item \textbf{0x8000 - 0xBFFF}: RAM oblast
        \item \textbf{0xC000 - 0xCFFF}: I/O oblast
        \item \textbf{0xD000 - 0xFFFF}: Speciální registry
    \end{itemize}
\end{itemize}
